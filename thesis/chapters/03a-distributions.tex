\section{Baseline – samples, dimensions and distances}
\label{section:distributions-experiment}

The goal of the first conducted experiment is to observe the behavior of selected outlierness measure and analyze their performance in the outlier detection task –~and to establish the baseline for more specific examinations, performed in further sections. The~effect of~three major parameters is considered: number of training samples $n$, dimension of feature vectors $d$ and the distance to out-of-distribution samples $h$; additionally utilized three different generator distribution functions to produce the data clusters – for even more versatile insight.

It should be noted that the organization of the simulated numerical study includes the ranges of $n$ and $d$ parameters values which are encountered in common OOD detection benchmarks for image and text recognition using Deep Learning models (chapter \ref{chapter:real-data}).


\subsection{Experiment organization}
\label{section:distributions-organization}

The experiment is organized as follows:
\vspace{-0.5\baselineskip}
\begin{itemize}
    \item First, 3 data clusters are generated.
          \begin{itemize}
              \item The set of \underline{training} data $T$, representing the in-distribution (ID) data, containing $n$ samples of dimension $d$, produced from a~chosen generator~$G$ (\textit{Gaussian}/\textit{MVN}, \textit{triangular} or \textit{uniform} distribution – that is located around the~center of~the coordinate system $\mu = [0, 0, \dots, 0]$ with spread of $\pm 1$).
              \item The set of \underline{known} data $K$, representing a testing dataset (another examples of~ID~data), generated from the same distribution as $T$, with a~fixed number of $1000$ samples. It is used to analyze the sensitivity of the detector, i.e., the~ability to properly recognize testing data as similar to the training data.
              \item The set of \underline{unknown} data $U$, representing out-of-distribution (OOD) data and consisting of a~fixed number of $1000$ samples, produced by the same generator as $T$, however with the distribution center shifted by the distance $h$~in~space (so the mean is at location $[\frac{h}{\sqrt{d}}, \frac{h}{\sqrt{d}}, \dots, \frac{h}{\sqrt{d}}]$). It is used to evaluate the specificity of the algorithm (i.e., proper detection of OOD samples).
          \end{itemize}
    \item The selected algorithm $OF$ (Outlier Factor) is fitted to the training dataset $T$.
    \item Next, the outlierness scores are calculated for each element of sets $T$, $K$ and $U$.
    \item The separability between clusters $K$ and $U$, using the selected $OF$, is analyzed by~calculating the Area Under the Receiver Operating Characteristic (AUROC).
    \item The classification of data from clusters $K$ and $U$ with respect to the dataset $T$ and outlierness measure $OF$ is performed, using the threshold value $t$ selected as the 95th and the 99th percentile of outlierness scores obtained for the cluster $T$.
\end{itemize}

Summarizing, the input parameters that vary in the experiment are: number of training samples $n$, dimension of feature space $d$, distance to the outliers $h$, outlierness measure $OF$ and the generator distribution $G$.

Additionally, for each combination of parameters, the experiment was repeated several times with various values of the generator seed $\xi$ (that affected the values within $T$, $K$ and $U$) to observe the variability of results.


\subsection{Experiment results – distribution properties}
\label{section:distributions-results-properties}

The aim of the first experiment was to analyze the behavior and usability of selected outlierness measures, discussed in section \ref{section:measures}, especially when applied in high-dimensional feature spaces, considering such factors as the number of training samples used to model the in-distribution data.

The study shows that various techniques used to model the in-distribution data, such~as~Euclidean distance (ED, section \ref{section:Euclidean}), Integrated Rank Weighted Depth (IRWD, section \ref{section:IRWD}), k-Nearest Neighbors (kNN, section \ref{section:kNN}) and Local Outlier Factor (LOF, section \ref{section:LOF}), have completely different properties. Although all of them can be considered as distance metrics, their output values are not directly comparable, e.g., the same element $v$ can obtain score $s=20$ using Mahalanobis distance (MD, section \ref{section:Mahalanobis}) and score $s=-0.65$ with Angle-Based Outlier Factor (ABOF, section \ref{section:ABOF}). Hence, selecting any arbitrary threshold value $t$ for distinguishing outliers, without additional analysis, is not universally possible.

Figure \ref{fig:histograms} presents example distributions of scores obtained for six different outlierness measures: ABOF, ED, IRWD, kNN, MD and LOF. First major difference between the techniques that can be noticed is in the ranges of values – as ED, kNN and MD are directly related to the spatial distances, the values are greater than for ABOF, IRWD and LOF, that rely on other quantities (variances scaled by distances, depth estimations with projections and comparison of neighbors' local reachability densities). The~results obtained for Standardized Euclidean distance (SED, section \ref{section:SEuclidean}) are nearly the same as for ED, due to the variance set to $1.0$ in the experiment, hence they are omitted and not presented in~figure~\ref{fig:histograms}.

It is worth to notice that from all considered measures only the IRWD is characterized by a limited range of the function value domain. For all other measures, there is either no upper limit or no lower limit.
\vspace{-0.5\baselineskip}
\begin{itemize}
    \item ABOF: $s \in \left(-\infty, 0.0\right]$,
    \item IRWD: $s \in \left[-0.5, 0.0\right]$,
    \item ED, kNN, LOF, MD, SED: $s \in \left[0.0, +\infty\right)$.
\end{itemize}
\vspace{-0.5\baselineskip}
Note that the score values of ABOF and IRWD are negative in this study, because in~implementation (Appendix \ref{chapter:source-code}) the returned values of formulas \ref{eq:abof} and \ref{eq:irwd} are inverted (multiplied by $-1$) to satisfy the criteria given by formula \ref{eq:open-set-classification} (section \ref{section:procedure}) –~i.e.~having greater values to indicate the outliers.

For most of measures (ED, IRWD, kNN, MD) the distributions appear symmetrical. However, in case of LOF the positive skew can be observed (the longer tail is on the right) in case of in-distribution data (train and known data in figure \ref{fig:histogram-lof}). Similarly, for ABOF the distributions are characterized by the negative skew (longer tail on the left side), especially in case of the distant out-of-distribution examples (unknown data in figure \ref{fig:histogram-abof}). This phenomenon is more clearly visible in figure \ref{fig:boxplots}.

The most important observation here is that for some algorithms, notably kNN and MD, the observed outlierness score values for known in-distribution samples (cluster $K$, i.e.~testing data) appear surprisingly distant from the values obtained for the training samples (cluster $T$). This means, that considering only the training samples' perspective (green barplots in figures \ref{fig:histogram-knn} and \ref{fig:histogram-mahalanobis}), most testing data coming from exactly the same distribution (blue barplots in the same figures) would be considered as out-of-distribution, i.e., outliers.

This phenomenon appears algorithm-specific and is observed regardless of generator distribution $G$, i.e., for $\textit{Gaussiann}$ (figures \ref{fig:histogram-knn} and \ref{fig:histogram-mahalanobis}), $\textit{Triangular}$ (figures \ref{fig:hists-knn-triangular} and \ref{fig:hists-md-triangular}) and $\textit{Uniform}$ (figures \ref{fig:hists-md-triangular} and \ref{fig:hists-md-uniform}). For both kNN and MD the effect intensifies in high-dimensional feature spaces (figure \ref{fig:hists-dimensions}). For MD the effect can be suppressed by increasing the number of training samples $n$, as visible in figure \ref{fig:hists-md-samples}, however for kNN the increased $n$ does not impact this phenomenon significantly (figures \ref{fig:hists-knn-500}, \ref{fig:hists-knn-1000} and \ref{fig:hists-knn-5000}). Inn case of kNN, the effect is reduced when greater number of $k$ neighbors is~considered (figure \ref{fig:hists-knn20-1000}).

The same effect can be observed for ED, SED and IRWD measures when the number of training samples $n$ is lower than the dimension of features space $d$, as visible in the figure \ref{fig:hists-extreme-bad} – just like for MD, increasing the number of training samples $n$ causes scores for in-distribution data (clusters $K$ and $T$) to overlap. It is caused by difficulty of obtaining accurate representation of a~cluster in high-dimensional features spaces, discussed further in section \ref{section:overlapping-experiment}. However, surprisingly, this effect was not observed in case of ABOF and LOF measures, even in case of extreme conditions, such as dimension of feature vectors $d = 5000$ and number of training samples $n = 50$, like shown in~the~figures~\ref{fig:hists-extreme-good}.

Despite that the training samples (cluster $T$) may appear distant from the known in-distribution samples (cluster $K$), in all discussed cases there is a~possibility to achieve a~good separation between in-distribution data and outliers (cluster $U$). Hence, the Receiver Operating Characteristic (ROC) curves presented in figure \ref{fig:rocs} look similar –~they present the relation between the sensitivity (True Positive Rate – TPR) and risk of type I error (False Positive Rate).

The ideal separation is reached in case of correct recognition of all in-distribution data without any spurious assignments of outliers – corresponding with top-left corner in ROC plot ($TPR = 1$, $FPR = 0$). The optimal threshold point marked in ROC plot is related to the threshold value that is closest to ideal situation (top-left ROC corner) – represented with the red cut-off vertical lines in figure \ref{fig:histograms}. The second marked threshold, TPR95, corresponds to a cut-off value for which the 95\% of in-distribution data were properly recognized.

The commonly used measure to evaluate the performance of classification model is the calculated Area Under the Receiver Operating Characteristic (AUROC) curve, with ideal value being $AUROC = 1.0$; any value $AUROC \leq 0.5$ means the classifier is worse than randomly performed assignments. In figure \ref{fig:rocs} all AUROCs are greater than $0.9$, indicating very well separation between clusters $K$ and $U$.

\begin{figure}[t]
    % StreamLit settings: width=5, height=3
    \centering
    \begin{subfigure}[b]{0.495\textwidth}
        \centering
        \caption{\small Angle-Based Outlier Factor}
        \includegraphics[width=\textwidth]{images/distributions/histograms/hist-distributions-dimension_250-samples_1000-distance_8-distribution_gaussian-model_ABOF-seed_0.pdf}
        \label{fig:histogram-abof}
    \end{subfigure}
    \hfill
    \begin{subfigure}[b]{0.495\textwidth}
        \centering
        \caption{\small Euclidean distance}
        \includegraphics[width=\textwidth]{images/distributions/histograms/hist-distributions-dimension_250-samples_1000-distance_8-distribution_gaussian-model_ED-seed_0.pdf}
        \label{fig:histogram-euclidean}
    \end{subfigure}
    \begin{subfigure}[b]{0.495\textwidth}
        \centering
        \caption{\footnotesize Integrated Rank Weighted Depth ({\scriptsize$n_{proj} = 10^3$})}
        \includegraphics[width=\textwidth]{images/distributions/histograms/hist-distributions-dimension_250-samples_1000-distance_8-distribution_gaussian-model_IRWD-1000-seed_0.pdf}
        \label{fig:histogram-irwd}
    \end{subfigure}
    \hfill
    \begin{subfigure}[b]{0.495\textwidth}
        \centering
        \caption{\small k-Nearest Neighbors ($k=10$)}
        \includegraphics[width=\textwidth]{images/distributions/histograms/hist-distributions-dimension_250-samples_1000-distance_8-distribution_gaussian-model_kNN-10-seed_0.pdf}
        \label{fig:histogram-knn}
    \end{subfigure}
    \begin{subfigure}[b]{0.495\textwidth}
        \centering
        \caption{\small Local Outlier Factor ($k=10$)}
        \includegraphics[width=\textwidth]{images/distributions/histograms/hist-distributions-dimension_250-samples_1000-distance_8-distribution_gaussian-model_LOF-10-seed_0.pdf}
        \label{fig:histogram-lof}
    \end{subfigure}
    \hfill
    \begin{subfigure}[b]{0.495\textwidth}
        \centering
        \caption{\small Mahalanobis distance}
        \includegraphics[width=\textwidth]{images/distributions/histograms/hist-distributions-dimension_250-samples_1000-distance_8-distribution_gaussian-model_MD-seed_0.pdf}
        \label{fig:histogram-mahalanobis}
    \end{subfigure}
    \caption{The distributions of outlierness scores obtained for various $OF$ measures (ABOF, ED, IRWD, kNN, LOF, MD). For all cases the same configuration of $T$, $K$ and $U$ clusters is~used – containing $n = 1000$ training samples, dimension of feature vectors $d = 250$, generated from $G = \textit{Gaussian}$ distribution, seed $\xi = 0$; outliers are shifted by~distance $h = 8$. In~some cases (kNN, MD) the results obtained for $K$ are~surprisingly~distant from results obtained for $T$.}
    \label{fig:histograms}
\end{figure}

\begin{figure}[t]
    % StreamLit settings: width=9, height=2
    \centering
    \begin{subfigure}[b]{\textwidth}
        \centering
        \caption{\small Angle-Based Outlier Factor}
        \includegraphics[width=\textwidth]{images/distributions/skew/box-distributions-dimension_250-samples_1000-distance_8-distribution_gaussian-model_ABOF-seed_0.pdf}
        \label{fig:box-abof}
    \end{subfigure}
    \begin{subfigure}[b]{\textwidth}
        \centering
        \caption{\small Euclidean distance}
        \includegraphics[width=\textwidth]{images/distributions/skew/box-distributions-dimension_250-samples_1000-distance_8-distribution_gaussian-model_ED-seed_0.pdf}
        \label{fig:box-ed}
    \end{subfigure}
    \begin{subfigure}[b]{\textwidth}
        \centering
        \caption{\small k-Nearest Neighbors ($k=10$)}
        \includegraphics[width=\textwidth]{images/distributions/skew/box-distributions-dimension_250-samples_1000-distance_8-distribution_gaussian-model_kNN-10-seed_0.pdf}
        \label{fig:box-knn}
    \end{subfigure}
    \begin{subfigure}[b]{\textwidth}
        \centering
        \caption{\small Local Outlier Factor ($k=10$)}
        \includegraphics[width=\textwidth]{images/distributions/skew/box-distributions-dimension_250-samples_1000-distance_8-distribution_gaussian-model_LOF-10-seed_0.pdf}
        \label{fig:box-lof}
    \end{subfigure}
    \caption{The boxplots of scores distributions obtained for selected $OF$ measures (ABOF, ED, kNN, LOF) calculated on $T$, $K$ and $U$ clusters – corresponding with selected histograms from the figure \ref{fig:histograms}. The positive skew is observed in case of LOF and negative skew in case of ABOF measure, while ED and kNN appear symmetric.}
    \label{fig:boxplots}
\end{figure}

\begin{figure}[t]
    % StreamLit settings: width=5, height=3
    \centering
    \begin{subfigure}[b]{0.495\textwidth}
        \centering
        \caption{\small k-Nearest Neighbors, $G = \textit{Gaussian}$}
        \includegraphics[width=\textwidth]{images/distributions/hists-Gen/hist-distributions-dimension_250-samples_1000-distance_8-distribution_gaussian-model_kNN-10-seed_0.pdf}
        \label{fig:hists-knn-gaussian}
    \end{subfigure}
    \hfill
    \begin{subfigure}[b]{0.495\textwidth}
        \centering
        \caption{\small Mahalanobis distance, $G = \textit{Gaussian}$}
        \includegraphics[width=\textwidth]{images/distributions/hists-Gen/hist-distributions-dimension_250-samples_1000-distance_8-distribution_gaussian-model_MD-seed_0.pdf}
        \label{fig:hists-md-gaussian}
    \end{subfigure}
    \begin{subfigure}[b]{0.495\textwidth}
        \centering
        \caption{\small k-Nearest Neighbors, $G = \textit{Triangular}$}
        \includegraphics[width=\textwidth]{images/distributions/hists-Gen/hist-distributions-dimension_250-samples_1000-distance_8-distribution_triangular-model_kNN-10-seed_0.pdf}
        \label{fig:hists-knn-triangular}
    \end{subfigure}
    \hfill
    \begin{subfigure}[b]{0.495\textwidth}
        \centering
        \caption{\small Mahalanobis distance, $G = \textit{Triangular}$}
        \includegraphics[width=\textwidth]{images/distributions/hists-Gen/hist-distributions-dimension_250-samples_1000-distance_8-distribution_triangular-model_MD-seed_0.pdf}
        \label{fig:hists-md-triangular}
    \end{subfigure}
    \begin{subfigure}[b]{0.495\textwidth}
        \centering
        \caption{\small k-Nearest Neighbors, $G = \textit{Uniform}$}
        \includegraphics[width=\textwidth]{images/distributions/hists-Gen/hist-distributions-dimension_250-samples_1000-distance_8-distribution_uniform-model_kNN-10-seed_0.pdf}
        \label{fig:hists-knn-uniform}
    \end{subfigure}
    \hfill
    \begin{subfigure}[b]{0.495\textwidth}
        \centering
        \caption{\small Mahalanobis distance, $G = \textit{Uniform}$}
        \includegraphics[width=\textwidth]{images/distributions/hists-Gen/hist-distributions-dimension_250-samples_1000-distance_8-distribution_uniform-model_MD-seed_0.pdf}
        \label{fig:hists-md-uniform}
    \end{subfigure}
    \caption{In case of kNN and MD, for high-dimensional feature vectors, the scores for known in-distribution data (cluster $K$) may not overlap with the scores obtained for the training samples (cluster $T$). This phenomenon is observed regardless of~chosen data distribution generator $G$: $\textit{Triangular}$ (shown in figures \ref{fig:hists-knn-triangular} and \ref{fig:hists-md-triangular}) or $\textit{Uniform}$ (figures \ref{fig:hists-knn-uniform} and \ref{fig:hists-md-uniform}) $\textit{Gaussian}$ (figures \ref{fig:histogram-knn} and \ref{fig:histogram-mahalanobis}). Other parameters are the same as for figure \ref{fig:histograms}: $n = 1000$, $d = 250$, $h = 8$, $\xi = 0$.}
    \label{fig:histograms-other-G}
\end{figure}

\begin{figure}[t]
    % StreamLit settings: width=9, height=2
    % Query: model = "MD" and dimension == 250 and 500 <= samples <= 5000 and distance == 8 and seed = 0 and distribution == "uniform"
    \centering
    \begin{subfigure}[b]{\textwidth}
        \centering
        \caption{\small Mahalanobis distance, training samples: $n = 500$}
        \includegraphics[width=\textwidth]{images/distributions/hists-md-samples/hist-distributions-dimension_250-samples_500-distance_8-distribution_uniform-model_MD-seed_0.pdf}
        \label{fig:hists-md-500}
    \end{subfigure}
    \begin{subfigure}[b]{\textwidth}
        \centering
        \caption{\small Mahalanobis distance, training samples: $n = 1000$}
        \includegraphics[width=\textwidth]{images/distributions/hists-md-samples/hist-distributions-dimension_250-samples_1000-distance_8-distribution_uniform-model_MD-seed_0.pdf}
        \label{fig:hists-md-1000}
    \end{subfigure}
    \begin{subfigure}[b]{\textwidth}
        \centering
        \caption{\small Mahalanobis distance, training samples: $n = 2500$}
        \includegraphics[width=\textwidth]{images/distributions/hists-md-samples/hist-distributions-dimension_250-samples_2500-distance_8-distribution_uniform-model_MD-seed_0.pdf}
        \label{fig:hists-md-2500}
    \end{subfigure}
    \begin{subfigure}[b]{\textwidth}
        \centering
        \caption{\small Mahalanobis distance, training samples: $n = 5000$}
        \includegraphics[width=\textwidth]{images/distributions/hists-md-samples/hist-distributions-dimension_250-samples_5000-distance_8-distribution_uniform-model_MD-seed_0.pdf}
        \label{fig:hists-md-5000}
    \end{subfigure}
    \caption{The distance between scores for known data (in-distribution, cluster $K$) and training examples (cluster $T$) gets smaller for MD when the training cluster $T$ contains more elements (parameter $n$). Note that the outliers (unknown examples, cluster $U$) are also moving closer to~$T$ (distances between medians $Q2_{T}$ and $Q2_{U}$: $\Delta{Q}_{n=500} \approx 13.31$ $\rightarrow$ $\Delta{Q}_{n=1000} \approx 8.19$ $\rightarrow$ $\Delta{Q}_{n=2500} \approx 6.22$ $\rightarrow$ $\Delta{Q}_{n=5000} \approx 5.68$), \\
    up to a~certain point –~when $K$ overlaps with $T$, then cluster $U$ no longer moves towards cluster $T$. Other distribution parameters involved: \\
    $d = 250$, $h = 8$, $G = \textit{Uniform}$, $\xi = 0$.}
    \label{fig:hists-md-samples}
\end{figure}

\begin{figure}[t]
    % StreamLit settings: width=9, height=2
    \centering
    \begin{subfigure}[b]{\textwidth}
        \centering
        \caption{\small k-Nearest Neighbors ($k=10$), training samples: $n = 500$}
        \includegraphics[width=\textwidth]{images/distributions/hists-knn-samples/hist-distributions-dimension_250-samples_500-distance_8-distribution_uniform-model_kNN-10-seed_0.pdf}
        \label{fig:hists-knn-500}
    \end{subfigure}
    \begin{subfigure}[b]{\textwidth}
        \centering
        \caption{\small k-Nearest Neighbors ($k=10$), training samples: $n = 1000$}
        \includegraphics[width=\textwidth]{images/distributions/hists-knn-samples/hist-distributions-dimension_250-samples_1000-distance_8-distribution_uniform-model_kNN-10-seed_0.pdf}
        \label{fig:hists-knn-1000}
    \end{subfigure}
    \begin{subfigure}[b]{\textwidth}
        \centering
        \caption{\small k-Nearest Neighbors ($k=10$), training samples: $n = 5000$}
        \includegraphics[width=\textwidth]{images/distributions/hists-knn-samples/hist-distributions-dimension_250-samples_5000-distance_8-distribution_uniform-model_kNN-10-seed_0.pdf}
        \label{fig:hists-knn-5000}
    \end{subfigure}
    \begin{subfigure}[b]{\textwidth}
        \centering
        \caption{\small k-Nearest Neighbors ($k=20$), training samples: $n = 1000$}
        \includegraphics[width=\textwidth]{images/distributions/hists-knn-samples/hist-distributions-dimension_250-samples_1000-distance_8-distribution_uniform-model_kNN-20-seed_0.pdf}
        \label{fig:hists-knn20-1000}
    \end{subfigure}
    \caption{Unlike for MD, increasing the number of training samples $n$ in cluster $T$ does not bring the cluster $K$ scores significantly closer to scores for cluster $T$. Scores obtained for outliers (cluster $U$) also remain unaffected by $n$. However, the results for~$K$ and $T$ start to overlap for larger values of $k$ (parameter of kNN algorithm). Experiment settings are the same as in figure \ref{fig:hists-md-samples} ({\small$d = 250$, $h = 8$, $G = \textit{Uniform}$, $\xi = 0$}).}
    \label{fig:hists-knn-samples}
\end{figure}

\begin{figure}[t]
    % StreamLit settings: width=9, height=2
    \centering
    \begin{subfigure}[b]{\textwidth}
        \centering
        \caption{\small k-Nearest Neighbors ($k=10$), dimension of feature vectors: $d = 100$}
        \includegraphics[width=\textwidth]{images/distributions/hists-dimensions/hist-distributions-dimension_100-samples_1000-distance_8-distribution_uniform-model_kNN-10-seed_0.pdf}
        \label{fig:hists-knn-d100}
    \end{subfigure}
    \begin{subfigure}[b]{\textwidth}
        \centering
        \caption{\small k-Nearest Neighbors ($k=10$), dimension of feature vectors: $d = 500$}
        \includegraphics[width=\textwidth]{images/distributions/hists-dimensions/hist-distributions-dimension_500-samples_1000-distance_8-distribution_uniform-model_kNN-10-seed_0.pdf}
        \label{fig:hists-knn-d500}
    \end{subfigure}
    \begin{subfigure}[b]{\textwidth}
        \centering
        \caption{\small Mahalanobis distance, dimension of feature vectors: $d = 100$}
        \includegraphics[width=\textwidth]{images/distributions/hists-dimensions/hist-distributions-dimension_100-samples_1000-distance_8-distribution_uniform-model_MD-seed_0.pdf}
        \label{fig:hists-md-d100}
    \end{subfigure}
    \begin{subfigure}[b]{\textwidth}
        \centering
        \caption{\small Mahalanobis distance, dimension of feature vectors: $d = 500$}
        \includegraphics[width=\textwidth]{images/distributions/hists-dimensions/hist-distributions-dimension_500-samples_1000-distance_8-distribution_uniform-model_MD-seed_0.pdf}
        \label{fig:hists-md-d500}
    \end{subfigure}
    \caption{The effect of distancing scores acquired for cluster $K$ from the scores obtained for cluster $T$, observed in case of kNN and MD measures, is stronger for increased dimensionality of feature vectors $d$. It can be noticed that for higher dimensions the scores values are also greater, as both the measures are based on spatial distances in features space, hence more feature vectors components contribute to~greater score values. Results visible in plots are obtained for experiment settings: $n = 1000$, $h = 8$, $G = \textit{Uniform}$, $\xi = 0$.}
    \label{fig:hists-dimensions}
\end{figure}

\begin{figure}[tb]
    \vspace{-2.0em}
    % StreamLit settings: width=9, height=2
    \centering
    \begin{subfigure}[b]{\textwidth}
        \centering
        \caption{\small Standardized Euclidean distance, dimension of feature vectors: $d = 100$}
        \includegraphics[width=\textwidth]{images/distributions/hists-sed-dimensions/hist-distributions-dimension_100-samples_1000-distance_8-distribution_uniform-model_SED-seed_0.pdf}
        \label{fig:hists-sed-100}
    \end{subfigure}
    \begin{subfigure}[b]{\textwidth}
        \centering
        \caption{\small Standardized Euclidean distance, dimension of feature vectors: $d = 500$}
        \includegraphics[width=\textwidth]{images/distributions/hists-sed-dimensions/hist-distributions-dimension_500-samples_1000-distance_8-distribution_uniform-model_SED-seed_0.pdf}
        \label{fig:hists-sed-500}
    \end{subfigure}
    \caption{For measures ABOF, IRWD, LOF, ED and SED, in typical conditions, $n \gtrsim d$, the separation between scores for in-distribution data (cluster $K$ and cluster $T$) is not observed, maintaining good overlapping even for a lower number of training samples $n$ than for MD. Experiment settings: $n = 1000$, $h = 8$, $G = \textit{Uniform}$, $\xi = 0$.}
    \label{fig:hists-sed-dimensions}
\end{figure}
\begin{figure}[tb]
    \vspace{-1.0em}
    % StreamLit settings: width=9, height=2
    \centering
    \begin{subfigure}[b]{\textwidth}
        \centering
        \caption{\small Angle-Based Outlier Factor, dimension: $d = 1000$, samples: $n = 50$}
        \includegraphics[width=\textwidth]{images/distributions/hists-extreme/hist-distributions-dimension_1000-samples_50-distance_8-distribution_uniform-model_ABOF-seed_0.pdf}
        \label{fig:hists-abof-n50}
    \end{subfigure}
    \begin{subfigure}[b]{\textwidth}
        \centering
        \caption{\small Local Outlier Factor ($k=10$), dimension: $d = 1000$, samples: $n = 50$}
        \includegraphics[width=\textwidth]{images/distributions/hists-extreme/hist-distributions-dimension_1000-samples_50-distance_8-distribution_uniform-model_LOF-10-seed_0.pdf}
        \label{fig:hists-lof-n50}
    \end{subfigure}
    \caption{In the performed study, ABOF and LOF were able to produce accurate representations even in case of significantly under-represented testing cluster –~obtained scores for $T$ and $K$ do overlap despite $n = 50$ training samples for $d = 1000$ dimension of feature vectors. Remaining experiment settings are: $h = 8$, $G = \textit{Uniform}$, $\xi = 0$.}
    \label{fig:hists-extreme-good}
    \vspace{-3.0em}
\end{figure}

\begin{figure}[t]
    % StreamLit settings: width=9, height=2
    \centering
    \begin{subfigure}[b]{\textwidth}
        \centering
        \caption{\small Integrated Rank Weighted Depth ({\scriptsize$n_{proj} = 10^3$}), dimension: $d = 500$, samples: $n = 50$}
        \includegraphics[width=\textwidth]{images/distributions/hists-extreme/hist-distributions-dimension_500-samples_50-distance_8-distribution_uniform-model_IRWD-1000-seed_0.pdf}
        \label{fig:hists-irwd-n50}
    \end{subfigure}
    \begin{subfigure}[b]{\textwidth}
        \centering
        \caption{\small Integrated Rank Weighted Depth ({\scriptsize$n_{proj} = 10^3$}), dimension: $d = 500$, samples: $n = 500$}
        \includegraphics[width=\textwidth]{images/distributions/hists-extreme/hist-distributions-dimension_500-samples_500-distance_8-distribution_uniform-model_IRWD-1000-seed_0.pdf}
        \label{fig:hists-irwd-n500}
    \end{subfigure}
    \begin{subfigure}[b]{\textwidth}
        \centering
        \caption{\small Standardized Euclidean distance, dimension: $d = 1000$, training samples: $n = 50$}
        \includegraphics[width=\textwidth]{images/distributions/hists-extreme/hist-distributions-dimension_1000-samples_50-distance_8-distribution_uniform-model_SED-seed_0.pdf}
        \label{fig:hists-sed-n50}
    \end{subfigure}
    \begin{subfigure}[b]{\textwidth}
        \centering
        \caption{\small Standardized Euclidean distance, dimension: $d = 1000$, training samples: $n = 1000$}
        \includegraphics[width=\textwidth]{images/distributions/hists-extreme/hist-distributions-dimension_1000-samples_1000-distance_8-distribution_uniform-model_SED-seed_0.pdf}
        \label{fig:hists-sed-n1000}
    \end{subfigure}
    \caption{For strongly under-represented training clusters, $n \ll d$, the effect of~not-overlapping between the scores for cluster $T$ and $K$ is observed in case of~IRWD, ED and SED measures. The effect vanishes when $n$ is not so low, yet it does not need to be as big as for MD to reach overlapping (settings: $h = 8$, $G = \textit{Uniform}$, $\xi = 0$).}
    \label{fig:hists-extreme-bad}
\end{figure}

\begin{figure}[t]
    % StreamLit settings: width=5, height=3
    \centering
    \begin{subfigure}[b]{0.495\textwidth}
        \centering
        \caption{\small Angle-Based Outlier Factor}
        \includegraphics[width=\textwidth]{images/distributions/rocs/roc-distributions-dimension_250-samples_1000-distance_8-distribution_gaussian-model_ABOF-seed_0.pdf}
        \label{fig:roc-abof}
    \end{subfigure}
    \hfill
    \begin{subfigure}[b]{0.495\textwidth}
        \centering
        \caption{\small Euclidean distance}
        \includegraphics[width=\textwidth]{images/distributions/rocs/roc-distributions-dimension_250-samples_1000-distance_8-distribution_gaussian-model_ED-seed_0.pdf}
        \label{fig:roc-euclidean}
    \end{subfigure}
    \begin{subfigure}[b]{0.495\textwidth}
        \centering
        \caption{\footnotesize Integrated Rank Weighted Depth ({\scriptsize$n_{proj} = 10^3$})}
        \includegraphics[width=\textwidth]{images/distributions/rocs/roc-distributions-dimension_250-samples_1000-distance_8-distribution_gaussian-model_IRWD-1000-seed_0.pdf}
        \label{fig:roc-irwd}
    \end{subfigure}
    \hfill
    \begin{subfigure}[b]{0.495\textwidth}
        \centering
        \caption{\small k-Nearest Neighbors ($k=10$)}
        \includegraphics[width=\textwidth]{images/distributions/rocs/roc-distributions-dimension_250-samples_1000-distance_8-distribution_gaussian-model_kNN-10-seed_0.pdf}
        \label{fig:roc-knn}
    \end{subfigure}
    \begin{subfigure}[b]{0.495\textwidth}
        \centering
        \caption{\small Local Outlier Factor ($k=10$)}
        \includegraphics[width=\textwidth]{images/distributions/rocs/roc-distributions-dimension_250-samples_1000-distance_8-distribution_gaussian-model_LOF-10-seed_0.pdf}
        \label{fig:roc-lof}
    \end{subfigure}
    \hfill
    \begin{subfigure}[b]{0.495\textwidth}
        \centering
        \caption{\small Mahalanobis distance}
        \includegraphics[width=\textwidth]{images/distributions/rocs/roc-distributions-dimension_250-samples_1000-distance_8-distribution_gaussian-model_MD-seed_0.pdf}
        \label{fig:roc-mahalanobis}
    \end{subfigure}
    \caption{The Receiver Operating Characteristic (ROC) curves obtained for various $OF$ measures (ABOF, ED, IRWD, kNN, LOF, MD). They show the sepearability between clusters $K$ and $U$ visible in corresponding plots from the~figure~\ref{fig:histograms}.
    Despite that some $OF$ methods represent $K$ as distant from $T$, they still can distinguish between $K$ and $U$ quite well, all acquiring high AUROC scores \\
    (the~area values of visible subplots are all greater than $0.9$).}
    \label{fig:rocs}
\end{figure}

\cleardoublepage{}

\subsection{Experiment results – effects of parameters}
\label{section:distributions-results-trends}

Figure \ref{fig:dimension} illustrates how the performance of outlierness measures is affected by the dimension of the feature vectors $d$, under fixed number of training samples $n = 2500$ and distance to outliers $h = 8$. The experiments shows that higher the feature space dimension $d$, the more challenging the comparison between data vectors is, as both classification accuracy and AUROC score decrease. The research was focused on ED, IRWD, kNN, LOF, MD and SED measures, omitting ABOF as computationally too expensive and impractical for usage (primarily due to $n$; although it was reaching promising top-scores for lower $n$ and $d$).

All of the analyzed measures $OF$ provide good separability of in-distribution data and outliers in lower dimensions – reaching AUROC value above $0.95$ for $d \leq 200$. In higher dimensions, the outliers distribution appear closer to the training data, so the obtained AUROC values are lower, decreasing exponentially with $d$. For dimension $d = 1000$ the AUROC reaches about ${\sim}0.85$ in case of ED and SED (plots overlap in figure \ref{fig:dimension-auroc}), ${\sim}0.84$ for kNN and LOF, ${\sim}0.78$ for MD and ${\sim}0.715$ in case of IRWD (results visible in subfigure \ref{fig:dimension-auroc}).

Although offering good separability, when the measures are involved in the classification task with respect to the training data only, not all $OF$s perform so well. Notably the kNN's performance falls drastically, reaching accuracy of ${\sim}0.5$ for $d \geq 250$. Similarly, the MD measure, after initially performing well ($d \leq 250$), shows gradual decay of accuracy in higher dimensions ($d \geq 500$). In both mentioned cases it is related to the lose of sensitivity, as visible in figure \ref{fig:dimension-sensitivity} – at some point all in-distribution data were recognized as outliers by kNN and MD (due to the same effect discussed in previous subsection \ref{section:distributions-results-properties} and visible in figure \ref{fig:hists-dimensions}).
% This is significant because, in the real-world scenario, where we would not know where the outliers are distributed, our OOD detection threshold should only rely on the distribution of the available training set. However, as shown in this example, this may lead to poor classification accuracy, as accuracy finally drops to 50\% (i.e., all data are classified as outliers).
% TODO \todo{significant}

Contrary, in case of ED, SED, IRWD and LOF the lowered accuracy in high-dimensions is related purely to the decaying specificity – some out-of-distribution data are seen as too close to the training data, such as in histograms in previous subsection \ref{section:distributions-results-properties} (subfigures \ref{fig:histogram-euclidean}, \ref{fig:histogram-irwd} and \ref{fig:histogram-lof}), hence spuriously considered as inliers.

Figure \ref{fig:samples} shows analogous research, analyzing the performance of outlierness measures $OF$ affected by the number of training samples $n$, having fixed dimension of the feature vectors $d = 750$ and distance to outliers $h = 8$. Surprisingly, no strong influence between the accuracy and separability (AUROC value) is observed – except for extremely underrepresented cases ($n < 100$ for $d = 750$, not show in the figure) or~Mahalanobis Distance measure.

The estimation of covariance matrix for MD in features space of dimension $d = 750$ requires at least $n \geq 750$ data samples. Hence, first reasonable result for MD visible in figure \ref{fig:samples-auroc} appears for $n = 1000$ –~AUROC value ${\sim}0.715$; for $n = 750$ samples the AUROC is ${\sim}0.5$. For $n = 10000$ samples the reaches close to top AUROC score ${\sim}0.85$.

The best separability in analyzed case is again observed for ED and SED measures (AUROC values ${\sim}0.86$), then kNN and LOF (AUROC values ${\sim}0.85$). The IRWD performed significantly worse, scoring AUROC value ${\sim}0.76$. Similarly like in previously examined case, the good AUROC score does not translate to good accuracy in the classification task with respect to the training data. Again, it is due to the zero TPR score (sensitivity) – all in-distribution data are incorrectly recognized as outliers, because the outlierness scores for testing set do not overlap the scores for training set (as visible in figure \ref{fig:hists-dimensions}). In case of MD, low accuracy is observed for up to $n \leq 2500$ samples, because of the same effect as for kNN, however by providing more training samples ($n \geq 5000$) the scores for in-distribution testing samples start to overlap with scores for training samples (effect visible in figure \ref{fig:hists-md-samples}), in the end obtaining one of the top accuracy for $n = 10000$. Yet, ED, SED and LOF reach similar accuracy for lower number of training samples $n$.

Finally, figure \ref{fig:distance} presents the performance of outlierness measures $OF$ as affected by the distance to outliers $h$, under fixed dimension of the feature vectors $d = 750$ and number of training samples $n = 2500$. Intuitively, the more distant the outliers are, the easier they are separable and detectable, like discussed in section \ref{section:near-far-ood} (Near OOD vs Far OOD). The relation is analogous as in first analyzed case – resulting in best separability for ED, SED, kNN and LOF, then for MD and worst for IRWD under given experiment configuration. Again, the worst possible accuracy is observed for kNN and MD, as for given $n$ and $d$ all the data are recognized as outliers (zero sensitivity). The ED, SED, IRWD and LOF initially do not recognize any outliers (zero specificity), until they are significantly distant $h > 4$.

The experiments were repeated for various distributions generators ($Gaussian$, $Triangular$, $Uniform$), however no significant differences were observed – both the classification and separability is easier (i.e., higher scores obtained) for a given set of parameters in case of the distributions with finite output domain ($G = Triangular$, $G = Uniform$), due to outliers appearing more distant, as seen in figure \ref{fig:histograms-other-G}, however the overall trends and behaviors of measures remain similar. Hence, only the results for $G = Gaussian$ distribution are presented in this section. Additionally it will make it easier comparable with following results in sections \ref{section:correlations-experiment} and \ref{section:variances-experiment}. All omitted results can be analyzed in the tooling described in appendix \ref{chapter:source-code}.

\begin{figure}[t]
    % Query: dimension >= 10 and dimension <= 1000
    \centering
    \begin{subfigure}[b]{0.9\textwidth}
        % StreamLit settings: width=9, height=4
        % X: [0.00, 1010.00]
        % Y: [0.69, 1.01]
        \centering
        \caption{\small Separability between in-distribution and out-of-distribution data}
        \includegraphics[width=\textwidth]{images/distributions/trends-d/trend-distributions-auroc(dimension)-samples_2500-distance_8-distribution_gaussian-model_ED,IRWD-1000,kNN-10,LOF-10,MD,SED-aggregated.pdf}
        \label{fig:dimension-auroc}
    \end{subfigure}
    \begin{subfigure}[b]{0.9\textwidth}
        % StreamLit settings: width=9, height=4
        % X: [0.00, 1010.00]
        % Y: [0.48, 1.00]
        \centering
        \caption{\small Classification with respect to the training data}
        \includegraphics[width=\textwidth]{images/distributions/trends-d/trend-distributions-accuracy_95(dimension)-samples_2500-distance_8-distribution_gaussian-model_ED,IRWD-1000,kNN-10,LOF-10,MD,SED-aggregated.pdf}
        \label{fig:dimension-accuracy}
    \end{subfigure}
    \begin{subfigure}[b]{0.495\textwidth}
        % StreamLit settings: width=5, height=3
        % X: [0.00, 1010.00]
        % Y: [-0.02, 1.05]
        \centering
        \caption{\small Correctly recognized in-distribution}
        \includegraphics[width=\textwidth]{images/distributions/trends-d/trend-distributions-sens_95(dimension)-samples_2500-distance_8-distribution_gaussian-model_ED,IRWD-1000,kNN-10,LOF-10,MD,SED-aggregated.pdf}
        \label{fig:dimension-sensitivity}
    \end{subfigure}
    \hfill
    \begin{subfigure}[b]{0.495\textwidth}
        % StreamLit settings: width=5, height=3
        % X: [0.00, 1010.00]
        % Y: [-0.02, 1.05]
        \centering
        \caption{\small Correctly recognized out-of-distribution}
        \includegraphics[width=\textwidth]{images/distributions/trends-d/trend-distributions-spec_95(dimension)-samples_2500-distance_8-distribution_gaussian-model_ED,IRWD-1000,kNN-10,LOF-10,MD,SED-aggregated.pdf}
        \label{fig:dimension-specificity}
    \end{subfigure}
    \caption{The performance of outlierness measures $OF$ as affected by the dimension of  the feature space $d$. The fixed parameters in the experiment are: number~of~training samples $n = 2500$, distance to outliers $h = 8$ and distribution $G = Gaussian$. The~results are aggregated for multiple generator seeds $\xi$ and~displayed as averages with~error~bars (standard deviation).}
    \label{fig:dimension}
    \vspace{-1.0em}
\end{figure}

\begin{figure}[t]
    % Query: samples >= 100 and samples <= 10000
    \centering
    \begin{subfigure}[b]{0.9\textwidth}
        % StreamLit settings: width=9, height=4
        % X: [95.00, 10500.00]
        % Y: [0.68, 0.88]
        \centering
        \caption{\small Separability between in-distribution and out-of-distribution data}
        \includegraphics[width=\textwidth]{images/distributions/trends-n/trend-distributions-auroc(samples)-dimension_750-distance_8-distribution_gaussian-model_ED,IRWD-1000,kNN-10,LOF-10,MD,SED-aggregated.pdf}
        \label{fig:samples-auroc}
    \end{subfigure}
    \begin{subfigure}[b]{0.9\textwidth}
        % StreamLit settings: width=9, height=4
        % X: [95.00, 10500.00]
        % Y: [0,49, 0.80]
        \centering
        \caption{\small Classification with respect to the training data}
        \includegraphics[width=\textwidth]{images/distributions/trends-n/trend-distributions-accuracy_95(samples)-dimension_750-distance_8-distribution_gaussian-model_ED,IRWD-1000,kNN-10,LOF-10,MD,SED-aggregated.pdf}
        \label{fig:samples-accuracy}
    \end{subfigure}
    \begin{subfigure}[b]{0.495\textwidth}
        % StreamLit settings: width=5, height=3
        % X: [95.00, 10500.00]
        % Y: [-0.02, 1.05]
        \centering
        \caption{\small Correctly recognized in-distribution}
        \includegraphics[width=\textwidth]{images/distributions/trends-n/trend-distributions-sens_95(samples)-dimension_750-distance_8-distribution_gaussian-model_ED,IRWD-1000,kNN-10,LOF-10,MD,SED-aggregated.pdf}
        \label{fig:samples-sensitivity}
    \end{subfigure}
    \hfill
    \begin{subfigure}[b]{0.495\textwidth}
        % StreamLit settings: width=5, height=3
        % X: [95.00, 10500.00]
        % Y: [-0.02, 1.05]
        \centering
        \caption{\small Correctly recognized out-of-distribution}
        \includegraphics[width=\textwidth]{images/distributions/trends-n/trend-distributions-spec_95(samples)-dimension_750-distance_8-distribution_gaussian-model_ED,IRWD-1000,kNN-10,LOF-10,MD,SED-aggregated.pdf}
        \label{fig:samples-specificity}
    \end{subfigure}
    \caption{The performance of outlierness measures $OF$ as affected by the number of training samples $n$. The fixed parameters in the experiment are: dimension of the~feature space $d = 750$, distance to outliers $h = 8$ and distribution $G = Gaussian$. The~results are aggregated for multiple generator seeds $\xi$ and displayed as averages with~error~bars (standard deviation).}
    \label{fig:samples}
    \vspace{-1.0em}
\end{figure}

\begin{figure}[t]
    \centering
    \begin{subfigure}[b]{0.9\textwidth}
        % StreamLit settings: width=9, height=4
        % X: [0.96, 16.60]
        % Y: [0.46, 1.02]
        \centering
        \caption{\small Separability between in-distribution and out-of-distribution data}
        \includegraphics[width=\textwidth]{images/distributions/trends-h/trend-distributions-auroc(distance)-dimension_750-samples_2500-distribution_gaussian-model_ED,IRWD-1000,kNN-10,LOF-10,MD,SED-aggregated.pdf}
        \label{fig:distance-auroc}
    \end{subfigure}
    \begin{subfigure}[b]{0.9\textwidth}
        % StreamLit settings: width=9, height=4
        % X: [0.96, 16.60]
        % Y: [0,48, 1.00]
        \centering
        \caption{\small Classification with respect to the training data}
        \includegraphics[width=\textwidth]{images/distributions/trends-h/trend-distributions-accuracy_95(distance)-dimension_750-samples_2500-distribution_gaussian-model_ED,IRWD-1000,kNN-10,LOF-10,MD,SED-aggregated.pdf}
        \label{fig:distance-accuracy}
    \end{subfigure}
    \begin{subfigure}[b]{0.495\textwidth}
        % StreamLit settings: width=5, height=3
        % X: [0.96, 16.60]
        % Y: [-0.02, 1.05]
        \centering
        \caption{\small Correctly recognized in-distribution}
        \includegraphics[width=\textwidth]{images/distributions/trends-h/trend-distributions-sens_95(distance)-dimension_750-samples_2500-distribution_gaussian-model_ED,IRWD-1000,kNN-10,LOF-10,MD,SED-aggregated (1).pdf}
        \label{fig:distance-sensitivity}
    \end{subfigure}
    \hfill
    \begin{subfigure}[b]{0.495\textwidth}
        % StreamLit settings: width=5, height=3
        % X: [0.96, 16.60]
        % Y: [-0.02, 1.05]
        \centering
        \caption{\small Correctly recognized out-of-distribution}
        \includegraphics[width=\textwidth]{images/distributions/trends-h/trend-distributions-spec_95(distance)-dimension_750-samples_2500-distribution_gaussian-model_ED,IRWD-1000,kNN-10,LOF-10,MD,SED-aggregated.pdf}
        \label{fig:distance-specificity}
    \end{subfigure}
    \caption{The performance of outlierness measures $OF$ as affected by the distance to outliers $h$. The fixed parameters in the experiment are: dimension of the~feature space $d = 750$, number of training samples $n = 2500$ and distribution $G = Gaussian$. The~results are aggregated for multiple generator seeds $\xi$ and displayed as averages with~error~bars (standard deviation).}
    \label{fig:distance}
    \vspace{-1.0em}
\end{figure}

\cleardoublepage{}

