\chapter{Selected personal achievements}
\label{chapter:achievements}

This chapter covers the author's personal background.


\section{List of scientific publications}
\label{section:publications}

Detailed list can be found in the database of the Wrocław University of Science and~Technology: \url{https://dona.pwr.edu.pl/szukaj/default.aspx?nrewid=600305}

\begin{itemize}
    \item
        \underline{Szymon Datko}, Kamil Szyc, Tomasz Walkowiak, Henryk Maciejewski,
        \textit{“How Characteristics of High-Dimensional Representations in Image and Text Recognition Impact the Performance of OOD Detectors“}, 2024.\\
        \textbf{Forthcoming}: under review.

    \item
        \underline{Szymon Datko}, Henryk Maciejewski, Tomasz Walkowiak,
        \textit{“Measures of Outlierness in High-Dimensional Data under Correlation of Features - with Application for Open-Set Classification“},
        proceedings of the Seventeenth International Conference on Dependability of Computer Systems, DepCoS-RELCOMEX, 2022.\\
        \textbf{DOI}: \url{https://doi.org/10.1007/978-3-031-06746-4_3}

    \item
        Tomasz Walkowiak, \underline{Szymon Datko}, Henryk Maciejewski,
        \textit{“Utilizing Local Outlier Factor for Open-Set Classification in High-Dimensional Data - Case Study Applied for Text Documents“},
        Intelligent Systems and Applications: proceedings of the 2019 Intelligent Systems Conference, IntelliSys, 2019.\\
        \textbf{DOI}: \url{https://doi.org/10.1007/978-3-030-29516-5_33}

    \item
        Tomasz Walkowiak, \underline{Szymon Datko}, Henryk Maciejewski,
        \textit{“Distance metrics in Open-Set Classification of Text Documents by Local Outlier Factor and Doc2Vec“},
        Advances and trends in artificial intelligence: from theory to practice: 32nd International Conference on Industrial, Engineering and Other Applications of Applied Intelligent Systems, IEA/AIE, 2019.\\
        \textbf{DOI}: \url{https://doi.org/10.1007/978-3-030-22999-3_10}

    \item
        Tomasz Walkowiak, \underline{Szymon Datko}, Henryk Maciejewski,
        \textit{“Low-dimensional classification of text documents“},
        Engineering in dependability of computer systems and networks: proceedings of the Fourteenth International Conference on Dependability of Computer Systems, DepCoS-RELCOMEX, 2019.\\
        \textbf{DOI}: \url{https://doi.org/10.1007/978-3-030-19501-4_53}

    \item
        Tomasz Walkowiak, \underline{Szymon Datko}, Henryk Maciejewski,
        \textit{“Open Set Subject Classification of Text Documents in Polish by Doc-to-Vec and Local Outlier Factor“},
        Artificial Intelligence and Soft Computing: 18th International Conference, ICAISC, 2019.\\
        \textbf{DOI}: \url{https://doi.org/10.1007/978-3-030-20915-5_41}

    \item
        Tomasz Walkowiak, \underline{Szymon Datko}, Henryk Maciejewski,
        \textit{“Reduction of dimensionality of feature vectors in subject classification of text documents“},
        Reliability and statistics in transportation and communication: selected papers from the 18th International Conference on Reliability and Statistics in Transportation and Communication, RelStat, 2018.\\
        \textbf{DOI}: \url{https://doi.org/10.1007/978-3-030-12450-2_15}

    \item
        Tomasz Walkowiak, \underline{Szymon Datko}, Henryk Maciejewski,
        \textit{“Bag-of-Words, Bag-of-Topics and Word-to-Vec Based Subject Classification of Text Documents in Polish - A Comparative Study“},
        Contemporary complex systems and their dependability: proceedings of the Thirteenth International Conference on Dependability and Complex Systems, DepCoS-RELCOMEX, 2018.\\
        \textbf{DOI}: \url{https://doi.org/10.1007/978-3-319-91446-6_49}

    \item
        Tomasz Walkowiak, \underline{Szymon Datko}, Henryk Maciejewski,
        \textit{“Feature Extraction in Subject Classification of Text Documents in Polish“},
        Artificial Intelligence and Soft Computing: 17th International Conference, ICAISC, 2018.\\
        \textbf{DOI}: \url{https://doi.org/10.1007/978-3-319-91262-2_40}

    \item
        Tomasz Walkowiak, \underline{Szymon Datko}, Henryk Maciejewski,
        \textit{“Algorithm Based on Modied Angle-Based Outlier Factor for Open-Set Classification of Text Documents“},
        Applied Stochastic Models in Business and Industry, ASMBI, 2018.\\
        \textbf{DOI}: \url{https://doi.org/10.1002/asmb.2388}
\end{itemize}


\section{List of conference speeches}

\begin{itemize}
    \item
        \underline{Szymon Datko}, Adrian Fusco Arnejo,
        \textit{“Dashboards as a Code: managing Grafana with Jsonnet“},
        OpenInfra Day Germany,
        May 2024, Berlin, Germany.\\
        Resources: {\footnotesize\url{https://github.com/adrianfusco/openinfra2024-dashboard-as-a-code}}

    \item
        \underline{Szymon Datko}, Ignacio Horcada Bernal,
        \textit{“Debugging Zuul jobs – now easier than ever, with Autoholds feature“},
        OpenInfra Summit Vancouver '23,
        June 2023, Vancouver, Canada.\\
        Recording: \url{https://www.youtube.com/watch?v=_GEaQGhZd9Y}

    \item
        Arie Bregman, \underline{Szymon Datko},
        \textit{“Combining Ansible and Terraform for CI – better together love story based on OVN-CI project“},
        Virtual Open Infrastructure Summit,
        November 2020.\\
        Recording: \url{https://www.youtube.com/watch?v=6D13rG0iawI}

    \item
        \underline{Szymon Datko}, Roman Dobosz,
        \textit{“Zuul, the Third – Throws Away Any Dirt! A quick-start introduction“},
        OpenInfra Summit Shanghai 2019,
        November 2019, Shanghai, China.\\
        Recording: \url{https://www.youtube.com/watch?v=_viUYriGdPw}

    \item
        \underline{Szymon Datko}, Roman Dobosz,
        \textit{“Does your Jenkins speak Gerrit? Functional testing for your pipelines, JobDSL and more“},
        OpenInfra Summit Shanghai 2019,
        November 2019, Shanghai, China.\\
        Recording: \url{https://www.youtube.com/watch?v=PmgIGnUrV5g}

    \item
        \underline{Szymon Datko}, Tomasz Walkowiak, Henryk Maciejewski,
        \textit{“Low-dimensional classification of text documents“},
        14th International Conference on Dependability of Computer Systems – DepCoS 2019,
        July 2019, Brunów Palace, Lwówek Śląski, Poland.

    \item
        \underline{Szymon Datko},
        \textit{“Zuul trzeci – wyrzuca zły kod na śmieci“},
        OpenInfra Days Poland 2019,
        June 2019, Kraków, Poland.

    \item
        \underline{Szymon Datko}, Roman Dobosz,
        \textit{“Testing Jenkins configuration changes – solidify your JCasC, Job DSL and Pipelines usage“},
        OpenInfra Summit Denver 2019,
        May 2019, Denver, USA.\\
        Recording: \url{https://www.youtube.com/watch?v=nvgeXkE65ac}

    \item
        Piotr Bielak, \underline{Szymon Datko},
        \textit{“Zuul v3 – more than a project gating system“},
        OpenInfra Wrocław Meetup \#11,
        March 2019, Wrocław, Poland.

    \item
        Piotr Bielak, \underline{Szymon Datko},
        \textit{“From messy XML to wonderful YAML and pretty Job DSL – an in-Jenkins migration story“},
        OpenStack Summit Berlin 2018,
        November 2018, Berlin, Germany.\\
        Recording: \url{https://www.youtube.com/watch?v=T7rD--ZOYRQ}

    \item
        \underline{Szymon Datko},
        \textit{“Aktualizacja OpenStacka – świeży raport z pola bitwy“},
        OpenStack Days Poland 2018,
        June 2018, Kraków, Poland.

    \item
        \underline{Szymon Datko},
        \textit{“CDS - simple, scalable, powerful CI/CD solution“},
        14th Linux Session,
        May 2017, Wrocław University of Science and Technology, Wrocław, Poland.\\
        Recording: \url{https://www.youtube.com/watch?v=RneLKacYVC0}

    \item
        \underline{Szymon Datko}, Henryk Maciejewski,
        \textit{“Outlier Detection in High-Dimensional Data – Applied for Open-Set Text Classification“},
        13th Workshop on Stochastic Models, Statistics and Their Applications,
        February 2017, Humboldt-Universität zu Berlin, Berlin, Germany.

    \item
        \underline{Szymon Datko},
        \textit{“Automate your life with Gitlab-CI“},
        Student Session 2016,
        August 2016, CERN European Organization for Nuclear Research, Geneva, Switzerland.\\
        Recording: \url{https://cds.cern.ch/record/2206413}
\end{itemize}


\section{Projects and grants}

\begin{itemize}
    \item
        Henryk Maciejewski (Principal Investigator), Tomasz Walkowiak (Co-Investigator), \underline{Szymon Datko} (Auxiliary investigator),
        \textit{Classification based on high-dimensional open-set data - with applications in Text
Mining},
        OPUS 11, National Science Centre, Poland (grant \textbf{2016/21/B/ST6/02159}).
\end{itemize}


\section{Other achievements}

\begin{itemize}
    \item Supported organization and running of the \textit{Konferencja Projektów Zespołowych} –~an~event for third-year students during which their team projects are presented to the audience of academic community members and invited industry representatives. It involved the coordination of contact between the companies and the University, registration of projects, preparing gifts and prizes for the conference participants, designing and ordering souvenir T-shirts, providing technical support during final projects presentations (8 editions, 2017–2024).\\
        URL: \url{https://kpz.pwr.edu.pl}

    \item Participated in the creation of a new course Machine Learning in Animations, including preparation of teaching materials, as part of the AI Tech project (2022).

    \item Assisted in the creation of a new specialization \textit{Grafika i Systemy Multimedialne} (ang. \textit{Graphics and Multimedia Systems}) in the field of Computer Science at~Wrocław University of Science and Technology – preparation of new courses cards, exam questions and teaching materials for the first and second degrees of~studies as part of the ZPR POWER project (2020).

    \item Prepared courses "Computer Graphics and Game Development Fundamentals" (2019) and "Introduction to DevOps and automation" (2020) for the TECHSummer international summer school, addressed to students from partner universities in~India.

    \item Organized a series of meetings and training on Linux systems administration and server infrastructure management for the members of the Section for Information Technology of the Student Government at Wrocław University of Science and Technology – as part of the Red Hat Academy program (2022–2023).

    \item Participated in the \textit{"PROJEKTOR"} voluntary work as part of the \textit{"IT for SHE"} program – organization of classes popularizing Computer Science and related fields (STEM – Science, Technology, Engineering and Math) among children and teenagers during summer camps, organized at the major Henryk Dobrzański's Primary School in Bircza, Poland (2017) and at the W. Witos's Public Primary School in Borek Strzeliński, Poland (2019).

    \item Conducted in total over 2500 hours of classes for students, teaching topics such~as: computer graphics, animations and simulations, design and programming of~computer games, software processing of images, scripting in~operating systems, acceleration of computations and diagnostics of digital circuits (2016-2024).

    \item Established an educational YouTube channel and prepared recorded introductions to~laboratory classes for students, from topics of computer graphics and shell scripts programming (2021–2024).\\
        URL: \url{https://www.youtube.com/c/SzymonDatko/videos}
\end{itemize}
