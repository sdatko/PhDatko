\chapter{Glossary}
\label{chapter:glossary}

{%
    \noindent
    \renewcommand{\arraystretch}{1.5}
    \begin{longtable}{>{\bfseries}l l p{0.75\textwidth}}
        ABOF
        &–&
        \textbf{A}ngle-\textbf{B}ased \textbf{O}utlier \textbf{F}actor
        \par
        \small
        A measure to quantity the similarity of the data.
        \newline
        See: section \ref{section:ABOF}
        \\

        BoW
        &–&
        \textbf{B}ag \textbf{o}f \textbf{W}ords
        \par
        \small
        A way to represent text documents as feature vectors.
        \\

        IAOF
        &–&
        \textbf{I}nterquartile \textbf{A}ngle-based \textbf{O}utlier \textbf{F}actor
        \par
        \small
        A measure to quantity the similarity of the data.
        \\

        IRWD
        &–&
        \textbf{I}ntegrated \textbf{R}ank \textbf{W}eighted \textbf{D}epth
        \par
        \small
        A measure to quantity the similarity of the data.
        \newline
        See: section \ref{section:IRWD}
        \\

        kNN
        &–&
        \textbf{k}-\textbf{N}earest \textbf{N}eighbors
        \par
        \small
        An algorithm for identifying closest neighbor points located in space.
        \newline
        See: section \ref{section:kNN}
        \\

        LOF
        &–&
        \textbf{L}ocal \textbf{O}utlier \textbf{F}actor
        \par
        \small
        A measure to quantity the similarity of the data.
        \newline
        See: section \ref{section:LOF}
        \\

        ML
        &–&
        \textbf{M}achine \textbf{L}earning
        \par
        \small
        A branch of Computer Science focused on imitating the way that humans learn, to produce tools to recognize and classify the data.
        \\

        NLP
        &–&
        \textbf{N}atural \textbf{L}anguage \textbf{P}rocessing
        \par
        \small
        A branch of Machine Learning focused on analysing text data.
        \\

        PCA
        &–&
        \textbf{P}rincipal \textbf{C}omponent \textbf{A}nalysis
        \par
        \small
        A method to reduce the dimensionality of feature vectors.
        \\

        RP
        &–&
        \textbf{R}andom \textbf{P}rojection
        \par
        \small
        A method to reduce the dimensionality of feature vectors.
        \\

        TF-IDF
        &–&
        \textbf{T}erm \textbf{F}requency – \textbf{I}nverse \textbf{D}ocument \textbf{F}requency
        \par
        \small
        A way to represent text documents as feature vectors.
        \\
    \end{longtable}
}
