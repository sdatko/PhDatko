\chapter{Introduction}
\label{chapter:introduction}


\section{Motivation – problem formulation}
\label{section:motivation}

The open-set classification is a task focused on identifying new data, i.e., data previously unseen and not related to any originally known category, during the classification process, performed by a~trained machine learning algorithm. Such task turns out to be important in any real-world implementation scenario \cite{Amodei-2016}, where samples from previously not considered classes can be fed into the system \cite{Hendrycks-2021}. Therefore detection of such samples which are out-of-distribution (OOD) with regard to the training data of the machine learning model, becomes the essential element for safety-critical applications of ML \cite{Hendrycks-2022}. However, while this task is already well established in literature \cite{Hodge-2004}\cite{Chandola-2009} in general, it is still not only open, but also surprisingly highly under-explored when applied to the high-dimensional data, such as feature vectors or representations generated by deep learning models for image and text recognition. Although many OOD detection methods for deep learning models have been proposed recently \cite{Geng-2021}, the~literature provides contradicting conclusions and recommendations \cite{Tajwar-2021}\cite{Yang-2022}, as performance of OOD detection methods is highly benchmark-dependent.

Over the past decade, a numerous approaches for solving the problem were proposed in publications \cite{Lee-2018}\cite{Hendrycks-2022-scaling}, aiming the performance improvements of deep learning models \cite{Wang-2022}\cite{Sun-2022}. Yet, although in presented benchmarks the results seem indicating that the task is successfully solved \cite{Yang-2022}, there are notable failures and mistakes observed in the real-world applications. While autonomous car reacting unexpectedly to a specifically painted t-shirt, or to a~drawing on a~billboard, may appear funny\footnote{\scriptsize\url{https://www.motortrend.com/news/can-a-t-shirt-stop-a-waymo-driverless-taxi-vehicle/}}, in fact it indicates serious flaws in the complex machine learning-based systems that are more and more widely used nowadays. The same car may therefore miss an obstacle or even worse –~a~living person, leading to a~seriously dangerous situation on the road.

The machine learning models are known to be affected by artificially introduced additional perturbations of inputs that may significantly change the models predictions \cite{Costa-2024} (adversarial attacks). Even a~modification of a~single image pixel can lead to a~spurious response from a~model \cite{Su-2019} (one pixel attack). Moreover, a~literature by Szyc et al. \cite{Szyc-2023} questions the reliability and methodological standards of the benchmarks available in lately published papers, as it turns out that the outcomes may be drastically different for various parameters and starting conditions selected when the model is trained. Therefore, depending on the publication chosen, contradictory results may be observed, leading to conflicting recommendations.

Recent publication by leading scientists in the field, Bengio et al. \cite{Bengio-2024}, draws attention to the problem and points out the recommendations, criticizing the fact that the vast majority of recent work focuses on raw benchmark results and less on ensuring the stability of methods and examinations of the phenomena that are key to the security and safety of real-world implementations. The authors suggest that there is a need for reorienting the research and development works more towards the improved robustness, explainability and transparency of the models. This includes the models' abilities to respond to new situations and to detect unexpected, previously unseen data samples. Hence, the task of open-set classification/OOD detection remains not only an open research problem in deep learning, but it is also an important practical problem to be resolved in real-world, safety-critical applications of deep learning methods.

This dissertation contributes to the field of open-set classification/OOD detection by deeply studying the selected \textit{post-hoc} OOD detection methods, implemented in high-dimensional feature spaces – providing new insight and increasing the understanding of the observed phenomena related to the flaws and strengths of different OOD detectors in high dimensional representations generated by different DL models. First, the numerical research on simulated data is conducted to identify the methods behaviors and properties in high dimensional data. Then, in the second part, the performance of OOD detectors (\textit{post-hoc} methods, operating in the feature space) is analyzed in the task of image recognition, utilizing various deep learning architectures, e.g., convolutional networks (CNN), vision transformers (ViT) and models trained on pair of images with descriptions (CLIP). It is shown that the results depend on the properties and characteristics of the representations, i.e., feature vectors generated by various generating models. A~number of measures and techniques are proposed to assess the risk of deep learning models implementations in real-world tasks due to the problem of OOD-generalization, i.e., susceptibility to errors in recognizing outliers, as well as to better calibrate the operating point of OOD detection methods, i.e., rejection threshold. Similar analyzes are conducted for text representations as well.


\section{Main contributions}
\label{section:contributions}

The key contributions of this dissertation are as follows:
\vspace{-0.5\baselineskip}
\begin{enumerate}
    \item Conducted a comprehensive study on the performance of selected \textit{post-hoc} OOD detection methods for outliers detection, measured more detail than in current literature (not only standard AUROC score, but also sensitivity and specificity), considering such factors as dimensions of feature vectors $d$, numbers of training samples $n$ and distance to outliers $h$ – to examine how well various methods can distinguish both training and testing samples from the~outliers; the values of $n$ and $d$ parameters reflected the characteristics of the training data occurring in the popular OOD detection benchmarks for image data, e.g., deep learning models based on the ImageNet dataset (section \ref{section:distributions-experiment}).
    \item Analyzed how the selected methods react to the presence of correlations in the data and identified group of methods that are performing well in such conditions and methods that are susceptible to increased errors in that case (section \ref{section:correlations-experiment}).
    \item Verified how the methods perform under non-uniform variance of features, identifying methods that effectively consider that parameter and are usable for unstandardized data (section \ref{section:variances-experiment}).
    \item Conducted a research on the ability of outliers detection methods for obtaining accurate model of the training data, identifying the methods' requirements to achieve and maintain such accuracy (section \ref{section:overlapping-experiment}).
    \item Analyzed the errors in the estimation of the covariance matrix values, based on the high-dimensional training data – and identified the impact of these errors on the performance of OOD/outlier detection methods that rely on those estimations (sections \ref{section:estimation-experiment}, \ref{section:Mahalanobis} and \ref{section:SEuclidean}).
    \item Proposed that the outliers detection techniques shall be compared not only by their ability to separate in-distribution and out-of-distribution data, but also on their classification performance (sensitivity, specificity) when calibrated on $\text{TPR} = 95\%$ with respect to the training data (section \ref{section:calibration}).
    \item Identified that some of the popular SoTA (\textit{State-of-The-Art}) methods may require calibration of threshold with the additional validation data for the effective realization of open-set classification in higher dimensions (section \ref{section:distributions-experiment}).
    \item Proposed that OOD detection research shall go beyond showing the overall AUROC measure, the current literature standard, in favor of per-class analysis –~presenting AUROC scores calculated per-class instead of a single average AUROC value provides additional insight into the safety risks of models' deployments due to classes with low OOD-generalization, which is especially important for safety-critical applications and shall be preferred in the OOD detection task benchmarks (section \ref{section:real-separability}).
    \item Identified that the performance of the outliers detection in real-world applications is~tightly related to the representation-generating model used for the data processing – images or text documents; this phenomenon has not enough attention in the literature, results are often presented for a~fixed representation (usually ResNet), attempting to form general conclusions that turn out not useful, as~different representations favor different outliers detection techniques and can be recommended for a~specific task configuration (sections \ref{section:real-separability} and \ref{section:real-classification}).
    \item Performed an analysis of the characteristics of the high-dimensional feature vectors produced from the various data representation-generating (deep learning) models for image and text recognition (section \ref{section:real-characteristics}).
    \item Shown that the popular data representation-generating models perform significantly different in terms of OOD-generalization. Hence, proposed a ranking of data representation models suitability for the outlier detection task (section \ref{section:real-recommendations}).
    \item Provided a Python library and a Python application for conducting a study of outlierness measures and performing an analysis of the experiments results (appendix \ref{chapter:source-code}).
\end{enumerate}

The author's existing contributions in the field are in addition as follows:
\vspace{-0.5\baselineskip}
\begin{enumerate}
    % general, all papers...
    \item Performed a study of various outlierness measures proposed in literature, such as Angle-Based Outlier Factor (ABOF), Local Outlier Factor (LOF), k-Nearest Neighbors (kNN) and Mahalanobis distance, in the task of detecting abnormal data (outliers) in high-dimensional feature spaces. \cite{Walkowiak-2018-asmbi}\cite{Walkowiak-2019-intellisys}\cite{Datko-2024}

    % 2018 ASMBI
    % Algorithm based on modified angle-based outlier factor for open-set classification of text documents
    \item Proposed a generic two-step procedure for open-set classification of text documents represented by high-dimensional feature vectors. \cite{Walkowiak-2018-asmbi}

    % 2018 ASMBI
    % Algorithm based on modified angle-based outlier factor for open-set classification of text documents
    \item Proposed a new method to quantity the outlierness in high-dimensional data –~IAOF (Interquartile Angle-based Outlier Factor). \cite{Walkowiak-2018-asmbi}

    % 2019 RelStat
    % Reduction of Dimensionality of Feature Vectors in Subject Classification of Text Documents
    \item Analyzed the usability of feature vectors reduction techniques, such as Principal Component Analysis (PCA) and Random Projection (RP), when applied in the task of open-set classification. \cite{Walkowiak-2019-relstat}

    % 2018 ICAISC
    % Feature Extraction in Subject Classification of Text Documents in Polish
    \item Conducted a study on different approaches to represent text documents with feature vectors, such as Bag of Words (BoW), Term Frequency – Inverse Document Frequency (TF-IDF), Word2Vec and fastText, in the task of subject classification of documents in Polish language. \cite{Walkowiak-2018-icaisc}

    % 2018 DepCoS
    % Bag-of-Words, Bag-of-Topics and Word-to-Vec Base Subject Classification of Text Documents in Polish – a comparative study
    \item Studied the performance of Natural Language Processing (NLP) technologies compared to approaches not-requiring the language knowledge, depending on the~number of training examples and feature vectors sizes. \cite{Walkowiak-2018-depcos}

    % 2019 ICAISC
    % Open Set Subject Classification of Text Documents in Polish by Doc-to-Vec and Local Outlier Factor
    \item Proposed a solution for performing open-set classification of text documents, involving the fastText algorithm and the Local Outlier Factor measure. \cite{Walkowiak-2019-icaisc}

    % 2019 DepCoS
    % Low-Dimensional Classification of Text Documents
    \item Analyzed how the dimensionality of feature vectors affects the classification performance for fastText algorithm and identified its ability to produce categories as focused projections in the feature-space. \cite{Walkowiak-2019-depcos}

    % 2019 IEA AIE
    % Distance Metrics in Open-Set Classification of Text Documents by Local Outlier Factor and Doc2Vec
    \item Studied how various distance metrics, such as euclidean and cosine distance, and the transformations of feature vectors, such as standardization and normalization, affect the results in the task of open-set classification of text documents. \cite{Walkowiak-2019-ieaaie}

    % 2019 IntelliSys
    % Utilizing Local Outlier Factor for Open-Set Classification in High-Dimensional Data - Case Study Applied for Text Documents
    \item Studied the problem of precision-recall relation and finding the threshold value in the open-set classification task – as reduction of incorrect assignments comes with a risk of rejecting also correctly labeled data. \cite{Walkowiak-2019-intellisys}

    % 2022 DepCoS
    % Measures of Outlierness in High-Dimensional Data under Correlation of Features – with Application for Open-Set Classification
    \item Measured how the correlation structure in data (both correlation strength and the~number of correlated variables), along with the dimensionality of feature vectors and the distance of outliers from typical data, affects the performance of outlierness measures. \cite{Datko-2022}

    % 2024
    % How Characteristics of High-Dimensional Representations in Image and Text Recognition Impact the Performance of OOD Detectors
    \item Conducted a research on the relation between the data representation models and the performance of outlierness measures in the task of outliers detection. \cite{Datko-2024}
\end{enumerate}

\clearpage{}


\section{Document organization}
\label{section:document-organisation}

This document is organized as follows.

Chapter \ref{chapter:related-work} covers the necessary background of outlier detection techniques, focusing on distinguishing main approaches already proposed in literature. A detailed description of selected \textit{post-hoc} methods is provided, as well as the required formalization and notation of the open-set classification task. The techniques of representing real-world data as feature vectors are also described.

Chapter \ref{chapter:simulations} presents the research conducted on simulated data that came from pseudorandom number generators (PRNGs). The performance of selected \textit{post-hoc} methods for outliers detection is analyzed, considering such factors as dimensions of feature vectors, numbers of training samples and distance to outliers – to examine how well various methods can distinguish both training and testing samples from the~outliers. Additionally, the effects of correlations presence in the data on the methods performances are analyzed, as well as the behaviors of the methods when the features are characterized by non-uniform variances, i.e., data are unstandardized.

Then, chapter~\ref{chapter:real-data} contains the results of research run on the real-world data. A~wide range of pre-trained representation algorithms is used to obtain the feature vectors representations of~text documents and image data, that are then examined for their potential in the~open-set classification task with respect to the training data. A~number of~significant differences between the representations are observed and discussed.

Finally, chapter \ref{chapter:summary} is the summary of the work, discussing the conclusions, impact on the field and potential future research to be conducted. Recommendations for improving OOD detection in high-dimensional data are provided.
