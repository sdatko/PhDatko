\chapter{Introduction}
\label{chapter:introduction}

\todo{Do zrobienia}  % TODO
\lipsum[1]


\section{Motivation – problem formulation}
\label{section:motivation}

\todo{Do zrobienia}  % TODO
\lipsum[1]


\section{Main contributions}
\label{section:contributions}

The key contributions of my research are as follows:
\begin{enumerate}
    % general, all papers...
    \item Performed a study of various outlierness measures proposed in literature, such as Angle-Based Outlier Factor (ABOF), Local Outlier Factor (LOF), k-Nearest Neighbors (kNN) and Mahalanobis distance, in the task of detecting abnormal data (outliers) in high-dimensional feature spaces.

    % 2018 ASMBI
    % Algorithm based on modified angle-based outlier factor for open-set classification of text documents
    \item Proposed a generic two-step procedure for open-set classification of text documents represented by high-dimensional feature vectors.

    % 2018 ASMBI
    % Algorithm based on modified angle-based outlier factor for open-set classification of text documents
    \item Proposed a new method to quantity the outlierness in high-dimensional data –~IAOF (Interquartile Angle-based Outlier Factor).

    % 2018 RelStat
    % Reduction of Dimensionality of Feature Vectors in Subject Classification of Text Documents
    \item Analysed the usability of feature vectors reduction techniques, such as Principal Component Analysis (PCA) and Random Projection (RP), when applied in the task of open-set classification.

    % 2018 ICAISC
    % Feature Extraction in Subject Classification of Text Documents in Polish
    \item Conducted a study on different approaches to represent text documents with feature vectors, such as Bag of Words (BoW), Term Frequency – Inverse Document Frequency (TF-IDF), Word2Vec and fastText, in the task of subject classification of documents in Polish language.

    % 2018 DepCoS
    % Bag-of-Words, Bag-of-Topics and Word-to-Vec Base Subject Classification of Text Documents in Polish – a comparative study
    \item Studied the performance of Natural Language Processing (NLP) technologies compared to approaches not-requiring the language knowledge, depending on the~number of training examples and feature vectors sizes.

    % 2019 ICAISC
    % Open Set Subject Classification of Text Documents in Polish by Doc-to-Vec and Local Outlier Factor
    \item Proposed a solution for performing open-set classification of text documents, involving the fastText algorithm and the Local Outlier Factor measure.

    % 2019 DepCoS
    % Low-Dimensional Classification of Text Documents
    \item Analysed how the dimensionality of feature vectors affects the classification performance for fastText algorithm and identified its ability to produce categories as focused projections in the feature-space.

    % 2019 IEA AIE
    % Distance Metrics in Open-Set Classification of Text Documents by Local Outlier Factor and Doc2Vec
    \item Studied how various distance metrics, such as euclidean and cosine distance, and the transformations of feature vectors, such as standardization and normalization, affect the results in the task of open-set classification of text documents.

    % 2019 IntelliSys
    % Utilizing Local Outlier Factor for Open-Set Classification in High-Dimensional Data - Case Study Applied for Text Documents
    \item Studied the problem of precision-recall relation and finding the threshold value in the open-set classification task – as reduction of incorrect assignments comes with a risk of rejecting also correctly labelled data.

    % 2022 DepCoS
    % Measures of Outlierness in High-Dimensional Data under Correlation of Features – with Application for Open-Set Classification
    \item Measured how the correlation structure in data (both correlation strength and the~number of correlated variables), along with the dimensionality of feature vectors and the distance of outliers from typical data, affects the performance of outlierness measures.

    % 2024
    % (in progress)
    \item Conducted a research on the representability of data – how well various outlier detection techniques can distinguish both training and testing samples from the~outliers to identify optimal threshold values.
\end{enumerate}


\section{Document organisation}
\label{section:document-organisation}

This document is organized as follows. Chapter \ref{chapter:related-work} covers the explanation of research context, introducing crucial terms and concepts. The main focus is on describing the outlierness measuring algorithms, analyzed during the performed work, as well as the techniques of representing real-world data as feature vectors. Chapter \ref{chapter:simulations} presents the research conducted on simulated data that came from pseudorandom number generators (PRNG). It shows the characteristics of analyzed measures. Chapter \ref{chapter:real-data} contains the results of research run on real-world data. It presents the discovered properties of feature vectors obtained from various representation techniques. Finally, chapter \ref{chapter:summary} is the summary of the work, discussing the conclusions, impact on the field and potential future research to be conducted.
